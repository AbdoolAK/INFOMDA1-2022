% Options for packages loaded elsewhere
\PassOptionsToPackage{unicode}{hyperref}
\PassOptionsToPackage{hyphens}{url}
%
\documentclass[
]{article}
\usepackage{amsmath,amssymb}
\usepackage{lmodern}
\usepackage{iftex}
\ifPDFTeX
  \usepackage[T1]{fontenc}
  \usepackage[utf8]{inputenc}
  \usepackage{textcomp} % provide euro and other symbols
\else % if luatex or xetex
  \usepackage{unicode-math}
  \defaultfontfeatures{Scale=MatchLowercase}
  \defaultfontfeatures[\rmfamily]{Ligatures=TeX,Scale=1}
\fi
% Use upquote if available, for straight quotes in verbatim environments
\IfFileExists{upquote.sty}{\usepackage{upquote}}{}
\IfFileExists{microtype.sty}{% use microtype if available
  \usepackage[]{microtype}
  \UseMicrotypeSet[protrusion]{basicmath} % disable protrusion for tt fonts
}{}
\makeatletter
\@ifundefined{KOMAClassName}{% if non-KOMA class
  \IfFileExists{parskip.sty}{%
    \usepackage{parskip}
  }{% else
    \setlength{\parindent}{0pt}
    \setlength{\parskip}{6pt plus 2pt minus 1pt}}
}{% if KOMA class
  \KOMAoptions{parskip=half}}
\makeatother
\usepackage{xcolor}
\usepackage[margin=1in]{geometry}
\usepackage{color}
\usepackage{fancyvrb}
\newcommand{\VerbBar}{|}
\newcommand{\VERB}{\Verb[commandchars=\\\{\}]}
\DefineVerbatimEnvironment{Highlighting}{Verbatim}{commandchars=\\\{\}}
% Add ',fontsize=\small' for more characters per line
\usepackage{framed}
\definecolor{shadecolor}{RGB}{248,248,248}
\newenvironment{Shaded}{\begin{snugshade}}{\end{snugshade}}
\newcommand{\AlertTok}[1]{\textcolor[rgb]{0.94,0.16,0.16}{#1}}
\newcommand{\AnnotationTok}[1]{\textcolor[rgb]{0.56,0.35,0.01}{\textbf{\textit{#1}}}}
\newcommand{\AttributeTok}[1]{\textcolor[rgb]{0.77,0.63,0.00}{#1}}
\newcommand{\BaseNTok}[1]{\textcolor[rgb]{0.00,0.00,0.81}{#1}}
\newcommand{\BuiltInTok}[1]{#1}
\newcommand{\CharTok}[1]{\textcolor[rgb]{0.31,0.60,0.02}{#1}}
\newcommand{\CommentTok}[1]{\textcolor[rgb]{0.56,0.35,0.01}{\textit{#1}}}
\newcommand{\CommentVarTok}[1]{\textcolor[rgb]{0.56,0.35,0.01}{\textbf{\textit{#1}}}}
\newcommand{\ConstantTok}[1]{\textcolor[rgb]{0.00,0.00,0.00}{#1}}
\newcommand{\ControlFlowTok}[1]{\textcolor[rgb]{0.13,0.29,0.53}{\textbf{#1}}}
\newcommand{\DataTypeTok}[1]{\textcolor[rgb]{0.13,0.29,0.53}{#1}}
\newcommand{\DecValTok}[1]{\textcolor[rgb]{0.00,0.00,0.81}{#1}}
\newcommand{\DocumentationTok}[1]{\textcolor[rgb]{0.56,0.35,0.01}{\textbf{\textit{#1}}}}
\newcommand{\ErrorTok}[1]{\textcolor[rgb]{0.64,0.00,0.00}{\textbf{#1}}}
\newcommand{\ExtensionTok}[1]{#1}
\newcommand{\FloatTok}[1]{\textcolor[rgb]{0.00,0.00,0.81}{#1}}
\newcommand{\FunctionTok}[1]{\textcolor[rgb]{0.00,0.00,0.00}{#1}}
\newcommand{\ImportTok}[1]{#1}
\newcommand{\InformationTok}[1]{\textcolor[rgb]{0.56,0.35,0.01}{\textbf{\textit{#1}}}}
\newcommand{\KeywordTok}[1]{\textcolor[rgb]{0.13,0.29,0.53}{\textbf{#1}}}
\newcommand{\NormalTok}[1]{#1}
\newcommand{\OperatorTok}[1]{\textcolor[rgb]{0.81,0.36,0.00}{\textbf{#1}}}
\newcommand{\OtherTok}[1]{\textcolor[rgb]{0.56,0.35,0.01}{#1}}
\newcommand{\PreprocessorTok}[1]{\textcolor[rgb]{0.56,0.35,0.01}{\textit{#1}}}
\newcommand{\RegionMarkerTok}[1]{#1}
\newcommand{\SpecialCharTok}[1]{\textcolor[rgb]{0.00,0.00,0.00}{#1}}
\newcommand{\SpecialStringTok}[1]{\textcolor[rgb]{0.31,0.60,0.02}{#1}}
\newcommand{\StringTok}[1]{\textcolor[rgb]{0.31,0.60,0.02}{#1}}
\newcommand{\VariableTok}[1]{\textcolor[rgb]{0.00,0.00,0.00}{#1}}
\newcommand{\VerbatimStringTok}[1]{\textcolor[rgb]{0.31,0.60,0.02}{#1}}
\newcommand{\WarningTok}[1]{\textcolor[rgb]{0.56,0.35,0.01}{\textbf{\textit{#1}}}}
\usepackage{graphicx}
\makeatletter
\def\maxwidth{\ifdim\Gin@nat@width>\linewidth\linewidth\else\Gin@nat@width\fi}
\def\maxheight{\ifdim\Gin@nat@height>\textheight\textheight\else\Gin@nat@height\fi}
\makeatother
% Scale images if necessary, so that they will not overflow the page
% margins by default, and it is still possible to overwrite the defaults
% using explicit options in \includegraphics[width, height, ...]{}
\setkeys{Gin}{width=\maxwidth,height=\maxheight,keepaspectratio}
% Set default figure placement to htbp
\makeatletter
\def\fps@figure{htbp}
\makeatother
\setlength{\emergencystretch}{3em} % prevent overfull lines
\providecommand{\tightlist}{%
  \setlength{\itemsep}{0pt}\setlength{\parskip}{0pt}}
\setcounter{secnumdepth}{-\maxdimen} % remove section numbering
\ifLuaTeX
  \usepackage{selnolig}  % disable illegal ligatures
\fi
\IfFileExists{bookmark.sty}{\usepackage{bookmark}}{\usepackage{hyperref}}
\IfFileExists{xurl.sty}{\usepackage{xurl}}{} % add URL line breaks if available
\urlstyle{same} % disable monospaced font for URLs
\hypersetup{
  pdftitle={P1\_MartinOkanik},
  pdfauthor={Martin Okanik},
  hidelinks,
  pdfcreator={LaTeX via pandoc}}

\title{P1\_MartinOkanik}
\author{Martin Okanik}
\date{2022-09-19}

\begin{document}
\maketitle

\hypertarget{loading-packages}{%
\section{Loading packages}\label{loading-packages}}

\begin{Shaded}
\begin{Highlighting}[]
\FunctionTok{library}\NormalTok{(ISLR)}

\FunctionTok{library}\NormalTok{(tidyverse)}

\FunctionTok{library}\NormalTok{(haven)}

\FunctionTok{library}\NormalTok{(readxl)}
\end{Highlighting}
\end{Shaded}

\hypertarget{data-types}{%
\section{Data types}\label{data-types}}

\begin{center}\rule{0.5\linewidth}{0.5pt}\end{center}

\begin{enumerate}
\def\labelenumi{\arabic{enumi}.}
\tightlist
\item
  \textbf{Run the following code in \texttt{R} and inspect their data
  types using the \texttt{class()} function. Try to guess beforehand
  what their types will be!}
\end{enumerate}

\begin{center}\rule{0.5\linewidth}{0.5pt}\end{center}

\begin{Shaded}
\begin{Highlighting}[]
\NormalTok{object\_1 }\OtherTok{\textless{}{-}} \DecValTok{1}\SpecialCharTok{:}\DecValTok{5}

\FunctionTok{class}\NormalTok{(object\_1) }\CommentTok{\# integer}
\end{Highlighting}
\end{Shaded}

\begin{verbatim}
## [1] "integer"
\end{verbatim}

\begin{Shaded}
\begin{Highlighting}[]
\NormalTok{object\_2 }\OtherTok{\textless{}{-}}\NormalTok{ 1L}\SpecialCharTok{:}\NormalTok{5L}

\FunctionTok{class}\NormalTok{(object\_2) }\CommentTok{\# integer}
\end{Highlighting}
\end{Shaded}

\begin{verbatim}
## [1] "integer"
\end{verbatim}

\begin{Shaded}
\begin{Highlighting}[]
\NormalTok{object\_3 }\OtherTok{\textless{}{-}} \StringTok{"{-}123.456"}

\FunctionTok{class}\NormalTok{(object\_3) }\CommentTok{\# character}
\end{Highlighting}
\end{Shaded}

\begin{verbatim}
## [1] "character"
\end{verbatim}

\begin{Shaded}
\begin{Highlighting}[]
\NormalTok{object\_4 }\OtherTok{\textless{}{-}} \FunctionTok{as.numeric}\NormalTok{(object\_2)}

\FunctionTok{class}\NormalTok{(object\_4) }\CommentTok{\# numeric}
\end{Highlighting}
\end{Shaded}

\begin{verbatim}
## [1] "numeric"
\end{verbatim}

\begin{Shaded}
\begin{Highlighting}[]
\NormalTok{object\_5 }\OtherTok{\textless{}{-}}\NormalTok{ letters[object\_1]}

\FunctionTok{class}\NormalTok{(object\_5) }\CommentTok{\# character}
\end{Highlighting}
\end{Shaded}

\begin{verbatim}
## [1] "character"
\end{verbatim}

\begin{Shaded}
\begin{Highlighting}[]
\NormalTok{object\_6 }\OtherTok{\textless{}{-}} \FunctionTok{as.factor}\NormalTok{(}\FunctionTok{rep}\NormalTok{(object\_5, }\DecValTok{2}\NormalTok{))}

\FunctionTok{class}\NormalTok{(object\_6) }\CommentTok{\# factor}
\end{Highlighting}
\end{Shaded}

\begin{verbatim}
## [1] "factor"
\end{verbatim}

\begin{Shaded}
\begin{Highlighting}[]
\NormalTok{object\_7 }\OtherTok{\textless{}{-}} \FunctionTok{c}\NormalTok{(}\DecValTok{1}\NormalTok{, }\DecValTok{2}\NormalTok{, }\DecValTok{3}\NormalTok{, }\StringTok{"4"}\NormalTok{, }\StringTok{"5"}\NormalTok{, }\StringTok{"6"}\NormalTok{)}

\FunctionTok{class}\NormalTok{(object\_7) }\CommentTok{\# character}
\end{Highlighting}
\end{Shaded}

\begin{verbatim}
## [1] "character"
\end{verbatim}

\begin{center}\rule{0.5\linewidth}{0.5pt}\end{center}

\begin{enumerate}
\def\labelenumi{\arabic{enumi}.}
\setcounter{enumi}{1}
\tightlist
\item
  \textbf{Convert \texttt{object\_7} back to a vector of numbers using
  the \texttt{as.numeric()} function}
\end{enumerate}

\begin{center}\rule{0.5\linewidth}{0.5pt}\end{center}

\begin{Shaded}
\begin{Highlighting}[]
\NormalTok{object\_7 }\OtherTok{\textless{}{-}} \FunctionTok{as.numeric}\NormalTok{(object\_7)}
\end{Highlighting}
\end{Shaded}

\begin{center}\rule{0.5\linewidth}{0.5pt}\end{center}

\begin{enumerate}
\def\labelenumi{\arabic{enumi}.}
\setcounter{enumi}{2}
\tightlist
\item
  \textbf{Make a list called \texttt{objects} containing object 1 to 7
  using the \texttt{list()} function.}
\end{enumerate}

\begin{center}\rule{0.5\linewidth}{0.5pt}\end{center}

\begin{Shaded}
\begin{Highlighting}[]
\NormalTok{objects }\OtherTok{\textless{}{-}} \FunctionTok{list}\NormalTok{(object\_1, object\_2, object\_3, object\_4, object\_5, object\_6,}

\NormalTok{                object\_7)}
\end{Highlighting}
\end{Shaded}

\begin{center}\rule{0.5\linewidth}{0.5pt}\end{center}

\begin{enumerate}
\def\labelenumi{\arabic{enumi}.}
\setcounter{enumi}{3}
\tightlist
\item
  \textbf{Make a data frame out of \texttt{object\_1},
  \texttt{object\_2}, and \texttt{object\_5} using the
  \texttt{data.frame()} function}
\end{enumerate}

\begin{center}\rule{0.5\linewidth}{0.5pt}\end{center}

\begin{Shaded}
\begin{Highlighting}[]
\NormalTok{dat }\OtherTok{\textless{}{-}} \FunctionTok{data.frame}\NormalTok{(}\AttributeTok{Var1 =}\NormalTok{ object\_1, }\AttributeTok{Var2 =}\NormalTok{ object\_2, }\AttributeTok{Var3 =}\NormalTok{ object\_5)}

\NormalTok{dat}
\end{Highlighting}
\end{Shaded}

\begin{verbatim}
##   Var1 Var2 Var3
## 1    1    1    a
## 2    2    2    b
## 3    3    3    c
## 4    4    4    d
## 5    5    5    e
\end{verbatim}

\begin{center}\rule{0.5\linewidth}{0.5pt}\end{center}

\begin{enumerate}
\def\labelenumi{\arabic{enumi}.}
\setcounter{enumi}{4}
\tightlist
\item
  \textbf{Useful functions for determining the size of a data frame are
  \texttt{ncol()} and \texttt{nrow()}. Try them out!}
\end{enumerate}

\begin{center}\rule{0.5\linewidth}{0.5pt}\end{center}

\begin{Shaded}
\begin{Highlighting}[]
\FunctionTok{ncol}\NormalTok{(dat)}
\end{Highlighting}
\end{Shaded}

\begin{verbatim}
## [1] 3
\end{verbatim}

\begin{Shaded}
\begin{Highlighting}[]
\FunctionTok{nrow}\NormalTok{(dat)}
\end{Highlighting}
\end{Shaded}

\begin{verbatim}
## [1] 5
\end{verbatim}

\hypertarget{loading-viewing-and-summarising-data}{%
\section{Loading, viewing, and summarising
data}\label{loading-viewing-and-summarising-data}}

\begin{center}\rule{0.5\linewidth}{0.5pt}\end{center}

\begin{enumerate}
\def\labelenumi{\arabic{enumi}.}
\setcounter{enumi}{5}
\tightlist
\item
  \textbf{Use the function \texttt{read\_csv()} to import the file
  ``data/googleplaystore.csv'' and store it in a variable called
  \texttt{apps}.}
\end{enumerate}

\begin{center}\rule{0.5\linewidth}{0.5pt}\end{center}

\begin{Shaded}
\begin{Highlighting}[]
\NormalTok{apps }\OtherTok{\textless{}{-}} \FunctionTok{read\_csv}\NormalTok{(}\StringTok{"data/googleplaystore.csv"}\NormalTok{)}
\end{Highlighting}
\end{Shaded}

\begin{verbatim}
## Rows: 10841 Columns: 13
## -- Column specification --------------------------------------------------------
## Delimiter: ","
## chr (11): App, Category, Size, Installs, Type, Price, Content Rating, Genres...
## dbl  (2): Rating, Reviews
## 
## i Use `spec()` to retrieve the full column specification for this data.
## i Specify the column types or set `show_col_types = FALSE` to quiet this message.
\end{verbatim}

\begin{Shaded}
\begin{Highlighting}[]
\NormalTok{apps}
\end{Highlighting}
\end{Shaded}

\begin{verbatim}
## # A tibble: 10,841 x 13
##    App   Categ~1 Rating Reviews Size  Insta~2 Type  Price Conte~3 Genres Last ~4
##    <chr> <chr>    <dbl>   <dbl> <chr> <chr>   <chr> <chr> <chr>   <chr>  <chr>  
##  1 "Pho~ ART_AN~    4.1     159 19M   10,000+ Free  0     Everyo~ Art &~ Januar~
##  2 "Col~ ART_AN~    3.9     967 14M   500,00~ Free  0     Everyo~ Art &~ Januar~
##  3 "U L~ ART_AN~    4.7   87510 8.7M  5,000,~ Free  0     Everyo~ Art &~ August~
##  4 "Ske~ ART_AN~    4.5  215644 25M   50,000~ Free  0     Teen    Art &~ June 8~
##  5 "Pix~ ART_AN~    4.3     967 2.8M  100,00~ Free  0     Everyo~ Art &~ June 2~
##  6 "Pap~ ART_AN~    4.4     167 5.6M  50,000+ Free  0     Everyo~ Art &~ March ~
##  7 "Smo~ ART_AN~    3.8     178 19M   50,000+ Free  0     Everyo~ Art &~ April ~
##  8 "Inf~ ART_AN~    4.1   36815 29M   1,000,~ Free  0     Everyo~ Art &~ June 1~
##  9 "Gar~ ART_AN~    4.4   13791 33M   1,000,~ Free  0     Everyo~ Art &~ Septem~
## 10 "Kid~ ART_AN~    4.7     121 3.1M  10,000+ Free  0     Everyo~ Art &~ July 3~
## # ... with 10,831 more rows, 2 more variables: `Current Ver` <chr>,
## #   `Android Ver` <chr>, and abbreviated variable names 1: Category,
## #   2: Installs, 3: `Content Rating`, 4: `Last Updated`
## # i Use `print(n = ...)` to see more rows, and `colnames()` to see all variable names
\end{verbatim}

\begin{center}\rule{0.5\linewidth}{0.5pt}\end{center}

\begin{enumerate}
\def\labelenumi{\arabic{enumi}.}
\setcounter{enumi}{6}
\tightlist
\item
  \textbf{Did any column get a variable type you did not expect?}
\end{enumerate}

\begin{center}\rule{0.5\linewidth}{0.5pt}\end{center}

\begin{Shaded}
\begin{Highlighting}[]
\CommentTok{\# (i) Size is \textless{}chr\textgreater{} because it is expressed in "M"}

\CommentTok{\# its dimensions would ideally be included in the name, e.g. "Size [M]"}

\CommentTok{\# (ii) Type should perhaps be a factor}

\CommentTok{\# (iii) Price should definitely be numeric}

\CommentTok{\# (iv) Content too}

\CommentTok{\# (v) Generes should be a factor, just like (ii)}

\CommentTok{\# (vi) Last Updated should use some suitable date{-}time format}
\end{Highlighting}
\end{Shaded}

\begin{center}\rule{0.5\linewidth}{0.5pt}\end{center}

\begin{enumerate}
\def\labelenumi{\arabic{enumi}.}
\setcounter{enumi}{7}
\tightlist
\item
  \textbf{Use the function \texttt{head()} to look at the first few rows
  of the \texttt{apps} dataset}
\end{enumerate}

\begin{center}\rule{0.5\linewidth}{0.5pt}\end{center}

\begin{Shaded}
\begin{Highlighting}[]
\FunctionTok{head}\NormalTok{(apps)}
\end{Highlighting}
\end{Shaded}

\begin{verbatim}
## # A tibble: 6 x 13
##   App    Categ~1 Rating Reviews Size  Insta~2 Type  Price Conte~3 Genres Last ~4
##   <chr>  <chr>    <dbl>   <dbl> <chr> <chr>   <chr> <chr> <chr>   <chr>  <chr>  
## 1 "Phot~ ART_AN~    4.1     159 19M   10,000+ Free  0     Everyo~ Art &~ Januar~
## 2 "Colo~ ART_AN~    3.9     967 14M   500,00~ Free  0     Everyo~ Art &~ Januar~
## 3 "U La~ ART_AN~    4.7   87510 8.7M  5,000,~ Free  0     Everyo~ Art &~ August~
## 4 "Sket~ ART_AN~    4.5  215644 25M   50,000~ Free  0     Teen    Art &~ June 8~
## 5 "Pixe~ ART_AN~    4.3     967 2.8M  100,00~ Free  0     Everyo~ Art &~ June 2~
## 6 "Pape~ ART_AN~    4.4     167 5.6M  50,000+ Free  0     Everyo~ Art &~ March ~
## # ... with 2 more variables: `Current Ver` <chr>, `Android Ver` <chr>, and
## #   abbreviated variable names 1: Category, 2: Installs, 3: `Content Rating`,
## #   4: `Last Updated`
## # i Use `colnames()` to see all variable names
\end{verbatim}

\begin{center}\rule{0.5\linewidth}{0.5pt}\end{center}

\begin{enumerate}
\def\labelenumi{\arabic{enumi}.}
\setcounter{enumi}{8}
\tightlist
\item
  \textbf{Repeat steps 5, 6, and 7 but now for ``data/students.xlsx''
  (NB: You'll need a function from the package \texttt{readxl}). Also
  try out the function \texttt{tail()} and \texttt{View()} (with a
  capital V).}
\end{enumerate}

\begin{center}\rule{0.5\linewidth}{0.5pt}\end{center}

\begin{Shaded}
\begin{Highlighting}[]
\NormalTok{students }\OtherTok{\textless{}{-}} \FunctionTok{read\_xlsx}\NormalTok{(}\StringTok{"data/students.xlsx"}\NormalTok{)}

\FunctionTok{head}\NormalTok{(students)}
\end{Highlighting}
\end{Shaded}

\begin{verbatim}
## # A tibble: 6 x 3
##   student_number grade programme
##            <dbl> <dbl> <chr>    
## 1        5117250  6.54 A        
## 2        6562582  7.57 A        
## 3        6000241  6.08 B        
## 4        4862862  7.71 A        
## 5        6561723  6.57 B        
## 6        5625916  7.90 B
\end{verbatim}

\begin{Shaded}
\begin{Highlighting}[]
\FunctionTok{tail}\NormalTok{(students)}
\end{Highlighting}
\end{Shaded}

\begin{verbatim}
## # A tibble: 6 x 3
##   student_number grade programme
##            <dbl> <dbl> <chr>    
## 1        5062746  7.43 B        
## 2        6560954  7.04 B        
## 3        6120285  6.71 A        
## 4        6553913  8.24 A        
## 5        4181101  5.62 B        
## 6        4639846  4.84 A
\end{verbatim}

\begin{Shaded}
\begin{Highlighting}[]
\CommentTok{\# student\_number should be integer}

\CommentTok{\# programme could be a factor}
\end{Highlighting}
\end{Shaded}

\begin{center}\rule{0.5\linewidth}{0.5pt}\end{center}

\begin{enumerate}
\def\labelenumi{\arabic{enumi}.}
\setcounter{enumi}{9}
\tightlist
\item
  \textbf{Create a summary of the three columns in the students dataset
  using the \texttt{summary()} function. What is the range of the grades
  achieved by the students?}
\end{enumerate}

\begin{center}\rule{0.5\linewidth}{0.5pt}\end{center}

\begin{Shaded}
\begin{Highlighting}[]
\FunctionTok{summary}\NormalTok{(students)}
\end{Highlighting}
\end{Shaded}

\begin{verbatim}
##  student_number        grade        programme        
##  Min.   :4011659   Min.   :4.844   Length:37         
##  1st Qu.:4862862   1st Qu.:6.390   Class :character  
##  Median :6000241   Median :7.151   Mode  :character  
##  Mean   :5686729   Mean   :6.991                     
##  3rd Qu.:6553913   3rd Qu.:7.573                     
##  Max.   :6997130   Max.   :9.291
\end{verbatim}

\hypertarget{data-transformation-with-dplyr-verbs}{%
\section{\texorpdfstring{Data transformation with \texttt{dplyr}
verbs}{Data transformation with dplyr verbs}}\label{data-transformation-with-dplyr-verbs}}

\hypertarget{filter}{%
\subsubsection{Filter}\label{filter}}

\begin{center}\rule{0.5\linewidth}{0.5pt}\end{center}

\begin{enumerate}
\def\labelenumi{\arabic{enumi}.}
\setcounter{enumi}{10}
\tightlist
\item
  \textbf{Look at the help pages for \texttt{filter()} (especially the
  examples) and show the students with a grade lower than 5.5}
\end{enumerate}

\begin{center}\rule{0.5\linewidth}{0.5pt}\end{center}

\begin{Shaded}
\begin{Highlighting}[]
\FunctionTok{filter}\NormalTok{(students, grade }\SpecialCharTok{\textless{}} \FloatTok{5.5}\NormalTok{)}
\end{Highlighting}
\end{Shaded}

\begin{verbatim}
## # A tibble: 3 x 3
##   student_number grade programme
##            <dbl> <dbl> <chr>    
## 1        6114656  5.16 A        
## 2        5265402  5.49 B        
## 3        4639846  4.84 A
\end{verbatim}

\begin{center}\rule{0.5\linewidth}{0.5pt}\end{center}

\begin{enumerate}
\def\labelenumi{\arabic{enumi}.}
\setcounter{enumi}{11}
\tightlist
\item
  \textbf{Show only the students with a grade higher than 8 from
  programme A}
\end{enumerate}

\begin{center}\rule{0.5\linewidth}{0.5pt}\end{center}

\begin{Shaded}
\begin{Highlighting}[]
\FunctionTok{filter}\NormalTok{(students, grade }\SpecialCharTok{\textgreater{}} \DecValTok{8}\NormalTok{, programme }\SpecialCharTok{==} \StringTok{"A"}\NormalTok{)}
\end{Highlighting}
\end{Shaded}

\begin{verbatim}
## # A tibble: 5 x 3
##   student_number grade programme
##            <dbl> <dbl> <chr>    
## 1        6352581  8.09 A        
## 2        6165611  8.02 A        
## 3        4133949  8.40 A        
## 4        4011659  8.94 A        
## 5        6553913  8.24 A
\end{verbatim}

\hypertarget{arrange}{%
\section{Arrange}\label{arrange}}

\begin{center}\rule{0.5\linewidth}{0.5pt}\end{center}

\begin{enumerate}
\def\labelenumi{\arabic{enumi}.}
\setcounter{enumi}{12}
\tightlist
\item
  \textbf{Sort the students dataset such that the students from
  programme A are on top of the data frame and within the programmes the
  highest grades come first.}
\end{enumerate}

\begin{center}\rule{0.5\linewidth}{0.5pt}\end{center}

\begin{Shaded}
\begin{Highlighting}[]
\FunctionTok{arrange}\NormalTok{(students, programme, }\SpecialCharTok{{-}}\NormalTok{grade)}
\end{Highlighting}
\end{Shaded}

\begin{verbatim}
## # A tibble: 37 x 3
##    student_number grade programme
##             <dbl> <dbl> <chr>    
##  1        4011659  8.94 A        
##  2        4133949  8.40 A        
##  3        6553913  8.24 A        
##  4        6352581  8.09 A        
##  5        6165611  8.02 A        
##  6        6997130  7.75 A        
##  7        4862862  7.71 A        
##  8        6562582  7.57 A        
##  9        4483974  7.46 A        
## 10        5128923  7.26 A        
## # ... with 27 more rows
## # i Use `print(n = ...)` to see more rows
\end{verbatim}

\hypertarget{select}{%
\section{Select}\label{select}}

\begin{center}\rule{0.5\linewidth}{0.5pt}\end{center}

\begin{enumerate}
\def\labelenumi{\arabic{enumi}.}
\setcounter{enumi}{13}
\tightlist
\item
  \textbf{Show only the \texttt{student\_number} and \texttt{programme}
  columns from the students dataset}
\end{enumerate}

\begin{center}\rule{0.5\linewidth}{0.5pt}\end{center}

\begin{Shaded}
\begin{Highlighting}[]
\FunctionTok{select}\NormalTok{(students, student\_number, programme)}
\end{Highlighting}
\end{Shaded}

\begin{verbatim}
## # A tibble: 37 x 2
##    student_number programme
##             <dbl> <chr>    
##  1        5117250 A        
##  2        6562582 A        
##  3        6000241 B        
##  4        4862862 A        
##  5        6561723 B        
##  6        5625916 B        
##  7        4096023 A        
##  8        6114656 A        
##  9        5265402 B        
## 10        5977188 B        
## # ... with 27 more rows
## # i Use `print(n = ...)` to see more rows
\end{verbatim}

\begin{Shaded}
\begin{Highlighting}[]
\NormalTok{students }\OtherTok{\textless{}{-}} \FunctionTok{mutate}\NormalTok{(students, }\AttributeTok{pass =}\NormalTok{ grade }\SpecialCharTok{\textgreater{}} \FloatTok{5.5}\NormalTok{)}

\NormalTok{students}
\end{Highlighting}
\end{Shaded}

\begin{verbatim}
## # A tibble: 37 x 4
##    student_number grade programme pass 
##             <dbl> <dbl> <chr>     <lgl>
##  1        5117250  6.54 A         TRUE 
##  2        6562582  7.57 A         TRUE 
##  3        6000241  6.08 B         TRUE 
##  4        4862862  7.71 A         TRUE 
##  5        6561723  6.57 B         TRUE 
##  6        5625916  7.90 B         TRUE 
##  7        4096023  5.92 A         TRUE 
##  8        6114656  5.16 A         FALSE
##  9        5265402  5.49 B         FALSE
## 10        5977188  7.26 B         TRUE 
## # ... with 27 more rows
## # i Use `print(n = ...)` to see more rows
\end{verbatim}

\begin{center}\rule{0.5\linewidth}{0.5pt}\end{center}

\begin{enumerate}
\def\labelenumi{\arabic{enumi}.}
\setcounter{enumi}{14}
\tightlist
\item
  \textbf{Use \texttt{mutate()} and \texttt{recode()} to change the
  codes in the programme column of the students dataset to their names.
  Store the result in a variable called \texttt{students\_recoded}}
\end{enumerate}

\begin{center}\rule{0.5\linewidth}{0.5pt}\end{center}

\begin{Shaded}
\begin{Highlighting}[]
\NormalTok{students\_recoded }\OtherTok{\textless{}{-}}\NormalTok{ students }\SpecialCharTok{\%\textgreater{}\%}

\FunctionTok{mutate}\NormalTok{(}\AttributeTok{programme =}\NormalTok{ programme }\SpecialCharTok{\%\textgreater{}\%} \FunctionTok{recode}\NormalTok{(}\StringTok{"A"} \OtherTok{=} \StringTok{"Science"}\NormalTok{, }\StringTok{"B"} \OtherTok{=} \StringTok{"Social Science"}\NormalTok{))}
\end{Highlighting}
\end{Shaded}

\hypertarget{data-processing-pipelines}{%
\section{Data processing pipelines}\label{data-processing-pipelines}}

\begin{center}\rule{0.5\linewidth}{0.5pt}\end{center}

\begin{enumerate}
\def\labelenumi{\arabic{enumi}.}
\setcounter{enumi}{15}
\tightlist
\item
  \textbf{Create a data processing pipeline that (a) loads the apps
  dataset, (b) parses the number of installs as `Downloads' variable
  using \texttt{mutate} and \texttt{parse\_number()}, (c) shows only
  apps with more than 500 000 000 downloads, (d) orders them by rating
  (best on top), and (e) shows only the relevant columns (you can choose
  which are relevant, but select at least the \texttt{Rating} and
  \texttt{Category} variables). Save the result under the name
  \texttt{popular\_apps}.}
\end{enumerate}

\begin{center}\rule{0.5\linewidth}{0.5pt}\end{center}

\begin{Shaded}
\begin{Highlighting}[]
\NormalTok{popular\_apps }\OtherTok{\textless{}{-}}

  \FunctionTok{read\_csv}\NormalTok{(}\StringTok{"data/googleplaystore.csv"}\NormalTok{) }\SpecialCharTok{\%\textgreater{}\%}

  \FunctionTok{mutate}\NormalTok{(}\AttributeTok{Downloads =} \FunctionTok{parse\_number}\NormalTok{(Installs)) }\SpecialCharTok{\%\textgreater{}\%}

  \FunctionTok{filter}\NormalTok{(Downloads }\SpecialCharTok{\textgreater{}} \FloatTok{5e8}\NormalTok{) }\SpecialCharTok{\%\textgreater{}\%}

  \FunctionTok{arrange}\NormalTok{(}\SpecialCharTok{{-}}\NormalTok{Rating) }\SpecialCharTok{\%\textgreater{}\%}

  \FunctionTok{select}\NormalTok{(App, Rating, Downloads, Category) }\SpecialCharTok{\%\textgreater{}\%}

  \FunctionTok{distinct}\NormalTok{(App, }\AttributeTok{.keep\_all =} \ConstantTok{TRUE}\NormalTok{)}
\end{Highlighting}
\end{Shaded}

\begin{verbatim}
## Rows: 10841 Columns: 13
## -- Column specification --------------------------------------------------------
## Delimiter: ","
## chr (11): App, Category, Size, Installs, Type, Price, Content Rating, Genres...
## dbl  (2): Rating, Reviews
## 
## i Use `spec()` to retrieve the full column specification for this data.
## i Specify the column types or set `show_col_types = FALSE` to quiet this message.
\end{verbatim}

\hypertarget{grouping-and-summarisation}{%
\section{Grouping and summarisation}\label{grouping-and-summarisation}}

\begin{center}\rule{0.5\linewidth}{0.5pt}\end{center}

\begin{enumerate}
\def\labelenumi{\arabic{enumi}.}
\setcounter{enumi}{16}
\tightlist
\item
  \textbf{Show the median, minimum, and maximum for the popular apps
  dataset you made in the previous assignment.}
\end{enumerate}

\begin{center}\rule{0.5\linewidth}{0.5pt}\end{center}

\begin{Shaded}
\begin{Highlighting}[]
\NormalTok{popular\_apps }\SpecialCharTok{\%\textgreater{}\%}

  \FunctionTok{summarise}\NormalTok{(}

    \AttributeTok{med =} \FunctionTok{median}\NormalTok{(Rating),}

    \AttributeTok{min =} \FunctionTok{min}\NormalTok{(Rating),}

    \AttributeTok{max =} \FunctionTok{max}\NormalTok{(Rating)}

\NormalTok{  )}
\end{Highlighting}
\end{Shaded}

\begin{verbatim}
## # A tibble: 1 x 3
##     med   min   max
##   <dbl> <dbl> <dbl>
## 1   4.3   3.7   4.5
\end{verbatim}

\begin{center}\rule{0.5\linewidth}{0.5pt}\end{center}

\begin{enumerate}
\def\labelenumi{\arabic{enumi}.}
\setcounter{enumi}{17}
\tightlist
\item
  \textbf{Add the median absolute deviation to the summaries you made
  before}
\end{enumerate}

\begin{center}\rule{0.5\linewidth}{0.5pt}\end{center}

\begin{Shaded}
\begin{Highlighting}[]
\NormalTok{popular\_apps }\SpecialCharTok{\%\textgreater{}\%}

  \FunctionTok{summarise}\NormalTok{(}

    \AttributeTok{med =} \FunctionTok{median}\NormalTok{(Rating),}

    \AttributeTok{min =} \FunctionTok{min}\NormalTok{(Rating),}

    \AttributeTok{max =} \FunctionTok{max}\NormalTok{(Rating),}

    \AttributeTok{mad =} \FunctionTok{mad}\NormalTok{(Rating)}

\NormalTok{  )}
\end{Highlighting}
\end{Shaded}

\begin{verbatim}
## # A tibble: 1 x 4
##     med   min   max   mad
##   <dbl> <dbl> <dbl> <dbl>
## 1   4.3   3.7   4.5 0.222
\end{verbatim}

\begin{center}\rule{0.5\linewidth}{0.5pt}\end{center}

\begin{enumerate}
\def\labelenumi{\arabic{enumi}.}
\setcounter{enumi}{18}
\tightlist
\item
  \textbf{Create a grouped summary of the ratings per category in the
  popular apps dataset.}
\end{enumerate}

\begin{center}\rule{0.5\linewidth}{0.5pt}\end{center}

\begin{Shaded}
\begin{Highlighting}[]
\NormalTok{    popular\_apps }\SpecialCharTok{\%\textgreater{}\%}

      \FunctionTok{group\_by}\NormalTok{(Category) }\SpecialCharTok{\%\textgreater{}\%}

      \FunctionTok{summarise}\NormalTok{(}

        \AttributeTok{med =} \FunctionTok{median}\NormalTok{(Rating),}

        \AttributeTok{min =} \FunctionTok{min}\NormalTok{(Rating),}

        \AttributeTok{max =} \FunctionTok{max}\NormalTok{(Rating),}

        \AttributeTok{mad =} \FunctionTok{mad}\NormalTok{(Rating)}

\NormalTok{      )}
\end{Highlighting}
\end{Shaded}

\begin{verbatim}
## # A tibble: 11 x 5
##    Category              med   min   max    mad
##    <chr>               <dbl> <dbl> <dbl>  <dbl>
##  1 BOOKS_AND_REFERENCE  3.9    3.9   3.9 0     
##  2 COMMUNICATION        4.2    4     4.4 0.222 
##  3 ENTERTAINMENT        4.3    4.3   4.3 0     
##  4 GAME                 4.5    4.5   4.5 0     
##  5 NEWS_AND_MAGAZINES   3.9    3.9   3.9 0     
##  6 PHOTOGRAPHY          4.5    4.5   4.5 0     
##  7 PRODUCTIVITY         4.4    4.4   4.4 0     
##  8 SOCIAL               4.2    4.1   4.5 0.148 
##  9 TOOLS                4.4    4.4   4.4 0     
## 10 TRAVEL_AND_LOCAL     4.25   4.2   4.3 0.0741
## 11 VIDEO_PLAYERS        4      3.7   4.3 0.445
\end{verbatim}

\hypertarget{final-exercise}{%
\section{Final exercise}\label{final-exercise}}

\begin{center}\rule{0.5\linewidth}{0.5pt}\end{center}

\begin{enumerate}
\def\labelenumi{\arabic{enumi}.}
\setcounter{enumi}{19}
\tightlist
\item
  \textbf{Create an interesting summary based on the Google play store
  apps dataset. An example could be ``do games get higher ratings than
  communication apps?''}
\end{enumerate}

\begin{center}\rule{0.5\linewidth}{0.5pt}\end{center}

\begin{Shaded}
\begin{Highlighting}[]
\CommentTok{\# which category has the highest and lowest spread (sd) of rankings?}

\CommentTok{\# what is the entire ranking?}

\CommentTok{\# loading, selecting relevant data, and some preliminary exploration}

\NormalTok{app\_data }\OtherTok{\textless{}{-}} \FunctionTok{read\_csv}\NormalTok{(}\StringTok{"data/googleplaystore.csv"}\NormalTok{) }\SpecialCharTok{\%\textgreater{}\%}

  \FunctionTok{select}\NormalTok{(App, Rating, Category) }\SpecialCharTok{\%\textgreater{}\%}

  \FunctionTok{na.omit}\NormalTok{()}
\end{Highlighting}
\end{Shaded}

\begin{verbatim}
## Rows: 10841 Columns: 13
## -- Column specification --------------------------------------------------------
## Delimiter: ","
## chr (11): App, Category, Size, Installs, Type, Price, Content Rating, Genres...
## dbl  (2): Rating, Reviews
## 
## i Use `spec()` to retrieve the full column specification for this data.
## i Specify the column types or set `show_col_types = FALSE` to quiet this message.
\end{verbatim}

\begin{Shaded}
\begin{Highlighting}[]
\FunctionTok{str}\NormalTok{(app\_data)}
\end{Highlighting}
\end{Shaded}

\begin{verbatim}
## tibble [9,366 x 3] (S3: tbl_df/tbl/data.frame)
##  $ App     : chr [1:9366] "Photo Editor & Candy Camera & Grid & ScrapBook" "Coloring book moana" "U Launcher Lite \x96 FREE Live Cool Themes, Hide Apps" "Sketch - Draw & Paint" ...
##  $ Rating  : num [1:9366] 4.1 3.9 4.7 4.5 4.3 4.4 3.8 4.1 4.4 4.7 ...
##  $ Category: chr [1:9366] "ART_AND_DESIGN" "ART_AND_DESIGN" "ART_AND_DESIGN" "ART_AND_DESIGN" ...
##  - attr(*, "na.action")= 'omit' Named int [1:1475] 24 114 124 127 130 131 135 164 181 186 ...
##   ..- attr(*, "names")= chr [1:1475] "24" "114" "124" "127" ...
\end{verbatim}

\begin{Shaded}
\begin{Highlighting}[]
\FunctionTok{summary}\NormalTok{(app\_data)}
\end{Highlighting}
\end{Shaded}

\begin{verbatim}
##      App                Rating        Category        
##  Length:9366        Min.   :1.000   Length:9366       
##  Class :character   1st Qu.:4.000   Class :character  
##  Mode  :character   Median :4.300   Mode  :character  
##                     Mean   :4.192                     
##                     3rd Qu.:4.500                     
##                     Max.   :5.000
\end{verbatim}

\begin{Shaded}
\begin{Highlighting}[]
\CommentTok{\# calculating sd of Rating per Category}

\NormalTok{sd\_by\_cat }\OtherTok{\textless{}{-}}\NormalTok{ app\_data }\SpecialCharTok{\%\textgreater{}\%}

  \FunctionTok{group\_by}\NormalTok{(Category) }\SpecialCharTok{\%\textgreater{}\%}

  \FunctionTok{summarise}\NormalTok{(}\AttributeTok{mean =} \FunctionTok{mean}\NormalTok{(Rating), }\AttributeTok{sd =} \FunctionTok{sd}\NormalTok{(Rating), }\AttributeTok{n =} \FunctionTok{n}\NormalTok{()) }\SpecialCharTok{\%\textgreater{}\%}

  \FunctionTok{arrange}\NormalTok{(sd)}

\NormalTok{sd\_by\_cat}
\end{Highlighting}
\end{Shaded}

\begin{verbatim}
## # A tibble: 33 x 4
##    Category            mean    sd     n
##    <chr>              <dbl> <dbl> <int>
##  1 EDUCATION           4.39 0.252   155
##  2 ENTERTAINMENT       4.13 0.303   149
##  3 WEATHER             4.24 0.331    75
##  4 PERSONALIZATION     4.34 0.353   314
##  5 ART_AND_DESIGN      4.36 0.358    62
##  6 BEAUTY              4.28 0.363    42
##  7 GAME                4.29 0.365  1097
##  8 HOUSE_AND_HOME      4.20 0.368    76
##  9 LIBRARIES_AND_DEMO  4.18 0.379    65
## 10 SHOPPING            4.26 0.405   238
## # ... with 23 more rows
## # i Use `print(n = ...)` to see more rows
\end{verbatim}

\hypertarget{conclusion}{%
\section{CONCLUSION:}\label{conclusion}}

according to this dataset, people tend to have:

\begin{enumerate}
\def\labelenumi{(\roman{enumi})}
\item
  the most differing opinions about health, bussiness and dating apps
\item
  the least differing opinions about education, weather and
  entertainment apps
\end{enumerate}

This is in line with the expectation that more important life decisions,
which also trigger increased emotions (dating, health etc.) provoke a
wider range of reactions, which might also carry over to the relevant
apps. Remarkably, weather apps were also spared from people's bad mood,
the dataset was probably not collected in Netherlands or the UK\ldots{}

The mean scores of various categories are different, but the spread of
standard deviations is significantly larger than the spread in mean
scores. The mean scores for categories with lower deviation seem to be
on average very slightly larger compared to those with higher deviation.
I cut my analysis before going to any significance tests\ldots{}

\end{document}
